\section{Background}\label{sec:background}

\subsection{Runtime Verification}
RV is a lightweight method of improving the integrity of deployed systems, and focuses on augmenting a system with additional monitoring functionality, in order to avoid unintended behavior at runtime.
This is accomplished through a variety of monitoring techniques, which check whether a system conforms to specifications based on traces or streams of data from the running system.

A wide range of RV methods have been developed over the years, offering a variety of different specification languages for expressing the desired behavior of the system including temporal logics such as Linear Temporal Logic (LTL)~\cite{pnueli1977ltl} and Signal Temporal Logic (STL)~\cite{donze2013stl} as well as domain-specific languages such as TeSSLa~\cite{convent2018tessla}.

RV also encompasses both passive monitoring techniques which focus on detecting errors without changing the behavior of the system, as well as more active techniques (also known as \emph{runtime enforcement}~\cite{falcone2010runtimeenforcement}) which aim to block or correct bad behaviors.

\subsection{NuRV}

NuRV~\cite{CimattiTT19a}\footnote{\url{https://nurv.fbk.eu}} is an extension of the nuXmv model checker for assumption-based LTL RV with partial observability and resets. Monitoring formulas are specified in LTL while assumptions are in SMV. Thanks to the assumption, the output of the monitor can be conclusive also if the formula contains future operators or if not all variables are observable.

The tool provides commands for online/offline monitoring and code generation into standalone monitor code. Using the online/offline monitor, LTL properties can be verified incrementally on finite traces from the system under scrutiny. The code generation currently supports C, C++, Common Lisp, and Java, and is extensible. Furthermore, from the same internal monitor automaton, the monitor can be generated into SMV modules, whose characteristics can be verified by Model Checking using nuXmv. Finally, the monitor can be generated as an FMU for co-simulation with the monitored system.

\subsection{TeSSLa}
The Temporal Stream-based Specification Language (TeSSLa)~\cite{convent2018tessla} framework\footnote{\url{https://tessla.io}} combines a language and a suite of tools designed for real-time verification of systems through data stream analysis. TeSSLa allows the declaration of input data types and the transformation of this data into new, derived streams by applying a series of defined operations. This approach enables effective monitoring of complex systems, ensuring accurate tracking and analysis without overly complex processes.

TeSSLa provides extensive libraries and supports the creation of macros. These macros allow users to define custom operations, simplifying the specification of complex behaviors and increasing the accessibility of the language. TeSSLa also supports the generation of detailed output streams, including statistical data with precise event timestamps, and allows integration with monitoring tools developed in modern programming languages such as Rust and Scala. Its integration with the metrics collection agent Telegraf~\cite{TT-Connector} contributes to its effectiveness in real-world applications.
At its core, TeSSLa's strength lies in its ability to map input data to meaningful outputs, which is essential for real-time system monitoring and informed decision-making in areas such as DT technologies.

\subsection{Digital Twins as a Service}
The DTaaS platform is a collaborative platform to build, use, and share DTs.
It is based-off a microservices architecture with dedicated software containers\footnote{Container is a software component at level-2 of the C4 model.} for DT assets, user workspaces, platform services, a front-end website, and service router.

One of the architectural principles used in the development of DTaaS is to conceive DTs as composed of reusable assets, which separate the functionality into their constituent parts.
Within DTaaS, data, models~\cite{Zambrano&22}, tools~\cite{qi2021enabling}, services~\cite{budiardjo2021digital,robles2023opentwins} and ready to use DTs~\cite{aziz2023distributed} have been identified as reusable assets.
The DT Assets software container provides an interface to perform create, reuse, update, and delete operations on the reusable stored within the DTaaS.

Users utilizing DTaaS have private workspaces in which they can build and use systems, from where they can access assets as a regular part of the filesystem.
All workspaces have internet access thereby enabling the integration of DTs running inside workspaces with external software systems.

Out-of-the-box, DTaaS supports multiple commonly used services across DTs and users.
The most commonly used are RabbitMQ and MQTT (communication), InfluxDB and MongoDB (data storage), and Grafana (data visualization).
Additionally, it is possible to host private services accessible to a selected number of users.
These services include the run-time services provided by TeSSLa and NuRV.

\subsection{FMI-based Co-simulation}
\label{sc:fmi}
Integrating verification methods early in development ensures system correctness from the start \cite{Prasad&21a}.
One approach is \textit{co-simulation}, which combines multiple simulation tools into a single simulation~\cite{Gomes2018,Kubler2000}.
Co-simulation is crucial for modeling complex systems co-developed by multiple organizations and systems whose complexity transcends the capabilities of any single simulation tool.

Interoperability between heterogeneous simulation tools is achieved using Functional Mock-up Units (FMUs) defined by the Functional Mock-up Interface (FMI) standard~\cite{FMI2014}.
An FMU encapsulates the behavior of a \emph{dynamic system}, whose state evolves according to \emph{evolution rules} and \emph{external stimuli}, into a discrete trajectory.
This allows complex behaviors to be represented modularly while protecting intellectual property.

Multiple FMUs are composed into a \emph{scenario} by coupling their input and output ports to represent the behavior of a complex system.
A \emph{coupling} signifies that the state of one FMU (the output) directly influences the state of another (the input).
A scenario is simulated using a co-simulation framework that interacts with the FMUs through their interface to advance them in lockstep and exchange values between the coupled ports.
