\section{Example integrations}\label{sec:examples}

Below five distinct examples showcasing the implementation of monitoring in the DTaaS platform are presented:  three utilizing NuRV and two utilizing TeSSLa.
For NuRV, the first example demonstrates a scenario where the components of a DT are validated before deployment.
This involves exporting the NuRV specification as an FMU and conducting co-simulations with the other components of the DT.
In the second example, the reusability of the FMU within a service-oriented architecture is demonstrated, enabling runtime verification on the deployed DT.
It listens to real-time sensing data sent by the PT through RabbitMQ to the DT, evaluating the truth value of LTL formulas.
In the third example, the NuRV specification is deployed on a standalone server, with its services exposed to the DT.
As a result, it is uncoupled with the DT instance.
For TeSSLa, the two examples demonstrate passive and active monitoring.
In passive monitoring, an alarm is raised in the event of a violation of the monitored conditions.
In contrast, active monitoring entails altering the system's behavior if a monitored condition is falsified.
An overview of the examples, their correspondence to the methodology, and their relative section are provided in \cref{tab:example_method_overview}.%
%
\begin{table}[ht]
    \centering
    \begin{tabular}{|l|l|l|}
        \hline
        \rowcolor{lightgray}
        \textbf{Example}         & \textbf{Methodology} & \textbf{Described in}     \\ \hline
        NuRV FMU monitor         & \methodone           & \cref{subsec:NuRVmoni}    \\ \hline
        NuRV FMU service monitor & \methodtwo           & \cref{subsec:NuRVsermoni} \\ \hline
        NuRV ORBit2 monitor      & \methodthree         & \cref{subsec:NuRVORBIT}   \\ \hline
        TeSSLa active monitor    & \methodthree         & \cref{subsec:TESLA1}      \\ \hline
        TeSSLa passive monitor   & \methodthree         & \cref{subsec:TESLA2}      \\ \hline
    \end{tabular}
    \caption{Overview of the examples that are presented and which methodology they correspond to. \morten{TODO: Follow IEEE style}}
    \label{tab:example_method_overview}
\end{table}