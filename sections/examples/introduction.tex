\section{Example integrations}\label{sec:examples}

Below five distinct examples showcasing the RV integration into the AS are presented: three utilizing NuRV and two utilizing TeSSLa.
For NuRV, the first example demonstrates a scenario where the components of a self-adaptive DT are validated before the system is deployed.
This involves exporting the NuRV specification as an FMU and conducting co-simulations with the other components of the system.
In the second example, the reusability of the FMU within a service-oriented architecture is demonstrated, enabling RV on the deployed system.
It listens to real-time sensing data sent by the Physical Twin (PT), i.e., the physical counterpart of the system, through RabbitMQ to the DT, evaluating the truth value of LTL formulas.
In the third example, the NuRV specification is deployed on a standalone server, with its services exposed to the DT.
As a result, it is uncoupled from the DT instance.
For TeSSLa, the two examples demonstrate passive and active monitoring.
In passive monitoring, an alarm is raised in the event of a violation of the monitored conditions.
In contrast, active monitoring entails altering the system's behavior if a monitored condition is falsified.